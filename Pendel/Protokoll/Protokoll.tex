\documentclass[twoside]{protokoll}
% ... wenn Sie nicht doppelseitig ausdrucken können, lassen Sie sie
% weg:
%\documentclass{protokoll}

% Setzen Sie hier "I" für den Teil 1 oder "II" für den Teil 2 ein:
\praktikum{I}
 
% Setzen Sie hier das jeweilige Versuchsgebiet ein (Mechanik,
% Akustik, Thermodynamik, Elektrizitätslehre, usw.):
\versuchsgebiet{(Gebiet)}

% Setzen Sie hier Ihren Namen und Ihre Matrikelnummer sowie Ihre
% Gruppennummer ein (ohne die Klammern):
\teilnehmer{Name 1. Teilnehmer, (Matrikelnummer)}
\teilnehmer{Name 2. Teilnehmer, (Matrikelnummer)}
\gruppe{(Gruppennummer)}

\begin{document}

% Ersetzen Sie den folgenden Text durch eine kurze Beschreibung der
% Versuchsziele.
\begin{versuchsziele}
  Kurze Beschreibung der Versuchsziele aller Teilversuche.
\end{versuchsziele}

\section{1M1 Messung der Erdbeschleunigung mit dem Pendel}

\begin{aufgabe}{Grundlagen}
  Knappe Beschreibung der theoretischen Grundlagen, Angabe der
  benötigten Formel(n), ohne Herleitung. Definition der verwendeten
  Formelzeichen.
\end{aufgabe}

Die simpleste Form eines Pendels ist das Mathematische Pendel.
Hierbei wird das Pendel als Punktmasse genähert und es gilt nur für kleine Auslenkungen, da dort $sin(\varphi) \approx \varphi$ gilt.
Die Rückstellkraft hängt von der Masse, der Erdbeschleunigung und der Auslenkungen des Pendels ab.
Mithilfe der kleinwinkel Näherung gilt für das Rückstellende Drehmoment:
\begin{equation}
    J \cdot \varphi'' = -m \cdot g \cdot l \cdot \varphi
\end{equation}
Für das Trägheitsmoment gilt $J = m_T \cdot l^2$.

% Bitte belassen Sie die Aufgabentexte in Ihrem Protokoll und beginnen
% Sie hier mit der Lösung der ersten Aufgabe:



\begin{aufgabe}{Versuchsaufbau und Versuchsdurchführung}
  Beschreibung des Versuchsaufbaus einschließlich beschrifteter Skizze
  oder Foto. Beschreibung der Versuchsdurchführung: Handgriffe an der
  Apparatur, verwendete Messwerterfassungseinstellungen, Messbereiche,
  Triggerbedingungen, etc.
\end{aufgabe}


\begin{aufgabe}{Rohdaten}
  Stellen Sie Ihre Rohdaten dar, tabellarisch für $l_p$ und $r_p$,
  grafisch für den Verlauf der Schwingung der Stange ohne Pendelkörper
  sowie der Pendelschwingung (für mindestens drei Messreihen).
\end{aufgabe}


\begin{aufgabe}{Auswertung}
  Bestimmen Sie für alle Messreihen die Periodendauer und ihre
  Messunsicherheit, indem Sie eine geeignete lineare Regression
  durchführen. Demonstrieren Sie, dass Sie das Trägheitsmoment der
  Stange kompensiert haben. Tabellieren Sie die Zwischenergebnisse
  der relevanten Größen samt ihrer Messunsicherheiten. Berechnen
  Sie die Erdbeschleunigung. Führen Sie eine Fehlerrechnung zur
  Bestimmung der Messunsicherheit durch. Geben Sie bei Ihrer
  Fehlerrechnung die Größe der Einzelbeiträge an, die zu dem
  Gesamtfehler führen. Diskutieren Sie, welcher Fehler den
  Gesamtfehler dominiert. Vergleichen Sie Ihr Endergebnis mit dem
  relevanten Literaturwert.
\end{aufgabe}

\end{document}
