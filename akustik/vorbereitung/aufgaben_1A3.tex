\section{1A3 Bestimmung des E-Moduls von Metallen}

\begin{aufgabe}{Grundlagen}
  Knappe Beschreibung der theoretischen Grundlagen, Angabe der
  benötigten Formel(n), ohne Herleitung. Definition der verwendeten
  Formelzeichen.
\end{aufgabe}

% Bitte belassen Sie die Aufgabentexte in Ihrem Protokoll und beginnen
% Sie hier mit der Lösung der ersten Aufgabe:



\begin{aufgabe}{Versuchsaufbau und Versuchsdurchführung}
  Beschreibung des Versuchsaufbaus einschließlich beschrifteter Skizze
  oder Foto. Beschreibung der Versuchsdurchführung: Handgriffe an der
  Apparatur, verwendete Messwerterfassungseinstellungen, Messbereiche,
  Triggerbedingungen, etc.
\end{aufgabe}


\begin{aufgabe}{Rohdaten}
  Stellen Sie die gemessenen Werte von $L$, $D$ und $M$ in
  tabellarischer Form dar. Visualisieren Sie einige typische
  Schwingungsverläufe sowie deren Fourierspektren in geeigneter
  Weise.
\end{aufgabe}


\begin{aufgabe}{Auswertung}
  Bestimmen Sie aus den aufgezeichneten Schwingungsvorgängen die
  Schwingungsfrequenzen und tabellieren Sie die erhaltenen
  Werte. Ermitteln Sie die Streuung der einzelnen $f_i$ und bestimmen
  Sie daraus den statistischen Fehler auf den Mittelwert. Erläutern
  und illustrieren Sie Ihr Vorgehen zur Bestimmung der systematischen
  Unsicherheiten auf die Frequenz. Diskutieren Sie die Unsicherheiten
  auf $L$, $D$ und $M$. Berechnen Sie den E-Modul und die Dichte der
  vorliegenden Stangen und die zugehörigen statistischen und
  systematischen Messunsicherheiten. Diskutieren Sie, welche
  Fehlerbeiträge den Gesamtfehler dominieren. Vergleichen Sie Ihre
  Ergebnisse mit einschlägigen Literaturwerten.
\end{aufgabe}



