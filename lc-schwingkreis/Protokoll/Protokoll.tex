%
% Vorlage für Versuchsprotokolle im Grundpraktikum an der RWTH Aachen
% -------------------------------------------------------------------
%
% Bitte verwenden Sie diese Vorlage zur Erstellung Ihrer
% Protokolle.
%
% Wenn möglich, drucken Sie Ihre Protokolle bitte doppelseitig
% aus. Wenn Sie das Protokoll doppelseitig ausdrucken, verwenden Sie
% die twoside Option ...
\documentclass[twoside]{protokoll}
% ... wenn Sie nicht doppelseitig ausdrucken können, lassen Sie sie
% weg:
%\documentclass{protokoll}

% Setzen Sie hier "I" für den Teil 1 oder "II" für den Teil 2 ein:
\praktikum{I}
 
% Setzen Sie hier das jeweilige Versuchsgebiet ein (Mechanik,
% Akustik, Thermodynamik, Elektrizitätslehre, usw.):
\versuchsgebiet{(Gebiet)}

% Setzen Sie hier Ihren Namen und Ihre Matrikelnummer sowie Ihre
% Gruppennummer ein (ohne die Klammern):
\teilnehmer{Name 1. Teilnehmer, (Matrikelnummer)}
\teilnehmer{Name 2. Teilnehmer, (Matrikelnummer)}
\gruppe{(Gruppennummer)}

\begin{document}

% Ersetzen Sie den folgenden Text durch eine kurze Beschreibung der
% Versuchsziele.
\begin{versuchsziele}
  Kurze Beschreibung der Versuchsziele aller Teilversuche.
\end{versuchsziele}

\section{1E3 Gekoppelte LC-Schwingkreise}

\begin{aufgabe}{Grundlagen}
  Knappe Beschreibung der theoretischen Grundlagen, Angabe der
  benötigten Formel(n), ohne Herleitung. Definition der verwendeten
  Formelzeichen.
\end{aufgabe}

% Bitte belassen Sie die Aufgabentexte in Ihrem Protokoll und beginnen
% Sie hier mit der Lösung der ersten Aufgabe:



\begin{aufgabe}{Versuchsaufbau und Versuchsdurchführung}
  Beschreibung des Versuchsaufbaus einschließlich
  Schaltbild. Beschreibung der Versuchsdurchführung: verwendete
  Messwerterfassungseinstellungen, Messbereiche, Triggerbedingungen,
  etc.
\end{aufgabe}


\begin{aufgabe}{Vorversuch: Charakterisierung der verwendeten Bauteile}
  Charakterisieren Sie die verwendeten Bauteile mit Digitalvoltmeter
  bzw. Messbrücke.
\end{aufgabe}


\begin{aufgabe}{Ungekoppelte Schwingung}
  Zeigen Sie den Verlauf der Kondensatorspannungen für den
  ungekoppelten Fall und bestimmen Sie die Schwingungsfrequenz samt
  Messunsicherheit. Vergleichen Sie sie mit Ihrer Erwartung.
\end{aufgabe}


\begin{aufgabe}{Gekoppelte Schwingung: Schwebung}
  Stellen Sie für einen festen Kopplungsgrad $k$ eine Schwebung
  dar. Zeigen Sie jeweils die Fourierspektren, bestimmen Sie die
  Eigenfrequenzen und daraus den Kopplungsgrad sowie dessen
  Messunsicherheit. Bestimmen Sie die zeitliche Verschiebung
  $\Delta{}t$ zwischen den beiden Einhüllenden der Schwebungen der
  beiden Kondensatoren und vergleichen Sie sie mit Ihrer
  Erwartung. Stellen Sie dar, wie sich das Frequenzspektrum mit dem
  Abstand der Spulen ändert. Untersuchen Sie die Verstärkung der
  Kopplung durch Verwendung eines Eisenkerns in den beiden Spulen.
\end{aufgabe}


\begin{aufgabe}{Gekoppelte Schwingung: gleich- und gegensinnige Anregung}
  Stellen Sie für einen festen Kopplungsgrad $k$ nacheinander die
  beiden Fundamentalschwingungen dar. Zeigen Sie jeweils die
  Fourierspektren und bestimmen Sie die zugängliche
  Eigenfrequenz. Berechnen Sie den Kopplungsgrad sowie dessen
  Messunsicherheit. Vergleichen Sie mit den Werten aus der Schwebung.
\end{aufgabe}
 
 
\end{document}
