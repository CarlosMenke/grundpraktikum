%
% Vorlage für Versuchsprotokolle im Grundpraktikum an der RWTH Aachen
% -------------------------------------------------------------------
%
% Bitte verwenden Sie diese Vorlage zur Erstellung Ihrer
% Protokolle.
%
% Wenn möglich, drucken Sie Ihre Protokolle bitte doppelseitig
% aus. Wenn Sie das Protokoll doppelseitig ausdrucken, verwenden Sie
% die twoside Option ...
\documentclass[twoside]{protokoll}
% ... wenn Sie nicht doppelseitig ausdrucken können, lassen Sie sie
% weg:
%\documentclass{protokoll}

% Setzen Sie hier "I" für den Teil 1 oder "II" für den Teil 2 ein:
\praktikum{I}
 
% Setzen Sie hier das jeweilige Versuchsgebiet ein (Mechanik,
% Akustik, Thermodynamik, Elektrizitätslehre, usw.):
\versuchsgebiet{(Akustik)}

% Setzen Sie hier Ihren Namen und Ihre Matrikelnummer sowie Ihre
% Gruppennummer ein (ohne die Klammern):
\teilnehmer{Maximilian Carlos Menke, 434170}
\teilnehmer{Andrea Roth, 428396}
\gruppe{A3}

\begin{document}
 
% Ersetzen Sie den folgenden Text durch eine kurze Beschreibung der
% Versuchsziele.
\begin{versuchsziele}
Ziel des Versuches ist, das Elastizitätzsmodul verschiedener Metallstäbe zu bestimmen.
Eine statische Messung liefert nur für dünne Drähte Ergebnisse weswegen das Elastizitätsmodul in unserem Fall mit einer Dynamischen Messung bestimmt wird. Hierfür erzeugen wir stehende Wellen in den Metallstäben. So können wir mit der Frequenz und der Wellenlänge die Phasengeschwindigkeit und somit auch das Elastizitätsmodul bestimmen. 
\end{versuchsziele}

\begin{aufgabe}{Versuchsaufbau und Versuchsdurchführung}
  Für die Bestimmung des E-Moduls messen wir die Schallwellen die mithilfe eines Gummihammers in dem Stab erzeugt wurden. Diese werden in ein Digitales Signal umgewandelt das wir mithilfe des Sensor CASSY darstellen können und analysieren. Hierzu benötigen wirfolgende Geräte.\\

\textbf{Benötigte Geräte:}
\begin{itemize}
\item Semsor CASSY
\item Universalmikrofon mit Stativstange
\item Sockel
\item Tischklemme
\item Metallstange 20cm

\end{itemize}
  
  
  
  Beschreibung des Versuchsaufbaus einschließlich beschrifteter Skizze
  oder Foto. Beschreibung der Versuchsdurchführung: Handgriffe an der
  Apparatur, verwendete Messwerterfassungseinstellungen, Messbereiche,
  Triggerbedingungen, etc.
\end{aufgabe}

%
% Hier werden die bearbeiteten Aufgaben ausgewählt, indem das
% Prozentzeichen (%) vor dem jeweiligen \input-Befehl entfernt
% wird. Die tex-Dateien können im Moodle in den entsprechenden
% Unterordnern bei den einzelnen Versuchsgebieten heruntergeladen
% werden. Achten Sie darauf, dass sich alle benötigten Dateien im
% selben Verzeichnis befinden, bevor Sie diese tex-Datei kompilieren.
%
% Die Bearbeitung der Aufgaben findet unmittelbar in der jeweiligen
% Datei ("aufgaben_xXy.z.tex") statt. Dort befinden sich Umgebungen
% der Form
%
% \begin{aufgabe}
%   ... (Hinweise) ... bzw.
%   ... (Aufgabentext) ...
% \end{aufgabe}
%
% Allgemeine Hinweise entfernen Sie bitte. Hinter dem
% versuchsspezifischen Aufgabentext und dem "\end{aufgabe}" beginnen
% Sie Ihre Lösung der Aufgabe.
%

%
% Grundpraktikum Teil I
%

% Mechanik
%\section{1M1 Messung der Erdbeschleunigung mit dem Pendel}

\begin{aufgabe}{Grundlagen}
  Knappe Beschreibung der theoretischen Grundlagen, Angabe der
  benötigten Formel(n), ohne Herleitung. Definition der verwendeten
  Formelzeichen.
\end{aufgabe}

% Bitte belassen Sie die Aufgabentexte in Ihrem Protokoll und beginnen
% Sie hier mit der Lösung der ersten Aufgabe:



\begin{aufgabe}{Versuchsaufbau und Versuchsdurchführung}
  Beschreibung des Versuchsaufbaus einschließlich beschrifteter Skizze
  oder Foto. Beschreibung der Versuchsdurchführung: Handgriffe an der
  Apparatur, verwendete Messwerterfassungseinstellungen, Messbereiche,
  Triggerbedingungen, etc.
\end{aufgabe}


\begin{aufgabe}{Rohdaten}
  Stellen Sie Ihre Rohdaten dar, tabellarisch für $l_p$ und $r_p$,
  grafisch für den Verlauf der Schwingung der Stange ohne Pendelkörper
  sowie der Pendelschwingung (für mindestens drei Messreihen).
\end{aufgabe}


\begin{aufgabe}{Auswertung}
  Bestimmen Sie für alle Messreihen die Periodendauer und ihre
  Messunsicherheit, indem Sie eine geeignete lineare Regression
  durchführen. Demonstrieren Sie, dass Sie das Trägheitsmoment der
  Stange kompensiert haben. Tabellieren Sie die Zwischenergebnisse
  der relevanten Größen samt ihrer Messunsicherheiten. Berechnen
  Sie die Erdbeschleunigung. Führen Sie eine Fehlerrechnung zur
  Bestimmung der Messunsicherheit durch. Geben Sie bei Ihrer
  Fehlerrechnung die Größe der Einzelbeiträge an, die zu dem
  Gesamtfehler führen. Diskutieren Sie, welcher Fehler den
  Gesamtfehler dominiert. Vergleichen Sie Ihr Endergebnis mit dem
  relevanten Literaturwert.
\end{aufgabe}




%\input{aufgaben_1M2.tex}
%\input{aufgaben_1M4.tex}

% Akustik
%\input{aufgaben_1A1.tex}
%\input{aufgaben_1A2.tex}
\section{1A3 Bestimmung des E-Moduls von Metallen}

\begin{aufgabe}{Grundlagen}
  Knappe Beschreibung der theoretischen Grundlagen, Angabe der
  benötigten Formel(n), ohne Herleitung. Definition der verwendeten
  Formelzeichen.
\end{aufgabe}

% Bitte belassen Sie die Aufgabentexte in Ihrem Protokoll und beginnen
% Sie hier mit der Lösung der ersten Aufgabe:



\begin{aufgabe}{Versuchsaufbau und Versuchsdurchführung}
  Beschreibung des Versuchsaufbaus einschließlich beschrifteter Skizze
  oder Foto. Beschreibung der Versuchsdurchführung: Handgriffe an der
  Apparatur, verwendete Messwerterfassungseinstellungen, Messbereiche,
  Triggerbedingungen, etc.
\end{aufgabe}


\begin{aufgabe}{Rohdaten}
  Stellen Sie die gemessenen Werte von $L$, $D$ und $M$ in
  tabellarischer Form dar. Visualisieren Sie einige typische
  Schwingungsverläufe sowie deren Fourierspektren in geeigneter
  Weise.
\end{aufgabe}


\begin{aufgabe}{Auswertung}
  Bestimmen Sie aus den aufgezeichneten Schwingungsvorgängen die
  Schwingungsfrequenzen und tabellieren Sie die erhaltenen
  Werte. Ermitteln Sie die Streuung der einzelnen $f_i$ und bestimmen
  Sie daraus den statistischen Fehler auf den Mittelwert. Erläutern
  und illustrieren Sie Ihr Vorgehen zur Bestimmung der systematischen
  Unsicherheiten auf die Frequenz. Diskutieren Sie die Unsicherheiten
  auf $L$, $D$ und $M$. Berechnen Sie den E-Modul und die Dichte der
  vorliegenden Stangen und die zugehörigen statistischen und
  systematischen Messunsicherheiten. Diskutieren Sie, welche
  Fehlerbeiträge den Gesamtfehler dominieren. Vergleichen Sie Ihre
  Ergebnisse mit einschlägigen Literaturwerten.
\end{aufgabe}




%\input{aufgaben_1A4.tex}


% Wärmelehre
%
%\input{aufgaben_1T1.tex}
%\input{aufgaben_1T2.tex}
%\input{aufgaben_1T3.tex}

% Elektrizitätslehre
%
%\section{1E1 Ladekurven eines Kondensators}

\begin{aufgabe}{Grundlagen}
  Knappe Beschreibung der theoretischen Grundlagen, Angabe der
  benötigten Formel(n), ohne Herleitung. Definition der verwendeten
  Formelzeichen.
\end{aufgabe}

% Bitte belassen Sie die Aufgabentexte in Ihrem Protokoll und beginnen
% Sie hier mit der Lösung der ersten Aufgabe:



\begin{aufgabe}{Vorversuch: Charakterisierung der verwendeten Bauteile}
  Charakterisieren Sie die verwendeten Bauteile mit Digitalvoltmeter
  und Messbrücke.
\end{aufgabe}


\begin{aufgabe}{Vorversuch: Bestimmung des Ohmschen Widerstands}
  Beschreiben Sie den Versuchsaufbau unter Verwendung eines
  Schaltbildes. Beschreiben Sie die Versuchsdurchführung unter Angabe
  der relevanten Messwerterfassungseinstellungen. Zeigen Sie
  exemplarisch die Häufigkeitsverteilungen für Spannung $U$ und Strom
  $I$ an einem Messpunkt und beschreiben Sie, wie Sie die
  statistischen Fehler für $U$ und $I$ bestimmen. Tragen Sie die
  Spannung $U$ gegen den Strom $I$ auf und bestimmen Sie aus der
  Steigung den Widerstand $R$ und seine statistische
  Messunsicherheit. Ermitteln Sie den systematischen Fehler mit der
  Verschiebemethode.
\end{aufgabe}


\begin{aufgabe}{Lade- und Entladekurven des Kondensators mit dem Oszilloskop}
  Beschreiben Sie den Versuchsaufbau unter Verwendung eines
  Schaltbildes. Beschreiben Sie die Versuchsdurchführung unter Angabe
  der relevanten Messwerterfassungseinstellungen. Zeigen Sie für den
  Lade- und den Entladevorgang jeweils ein Bild der gemessenen
  Spannungsverläufe auf dem Oszilloskopschirm. Lesen Sie mithilfe der
  Cursor jeweils die Zeitkonstante ab. Berechnen Sie den gewichteten
  Mittelwert der Zeitkonstanten. Berechnen Sie daraus die Kapazität
  und geben Sie sie mit statistischer und systematischer
  Messunsicherheit an.
\end{aufgabe}


\begin{aufgabe}{Lade- und Entladekurven des Kondensators mit Cassy}
  Zeigen Sie für den Lade- und den Entladevorgang jeweils ein Bild des
  Spannungsverlaufs am Kondensator und des Lade- bzw.~Entladestroms.
  Korrigieren Sie erforderlichenfalls den Spannungs- und/oder
  Strom-Offset. Transformieren Sie die Rohdaten geeignet in
  logarithmische Größen und bestimmen Sie mittels linearer Regression
  die Zeitkonstante. Beschreiben Sie, wie Sie die Messunsicherheiten
  behandeln. Berechnen Sie das gewichtete Mittel und geben Sie Ihr
  Endergebnis für die Kapazität mit statistischer und systematischer
  Messunsicherheit an.
\end{aufgabe}


\begin{aufgabe}{Zusammenfassung und Diskussion}
  Fassen Sie Ihre Ergebnisse zu den Kapazitäten zusammen und
  vergleichen Sie sie untereinander, mit den Messungen aus dem
  Vorversuch und mit den Herstellerangaben (Toleranz
  $\SI{5}{\percent}$).  
\end{aufgabe}


%\input{aufgaben_1E2.tex}
%\section{1E3 Gekoppelte LC-Schwingkreise}

\begin{aufgabe}{Grundlagen}
  Knappe Beschreibung der theoretischen Grundlagen, Angabe der
  benötigten Formel(n), ohne Herleitung. Definition der verwendeten
  Formelzeichen.
\end{aufgabe}

% Bitte belassen Sie die Aufgabentexte in Ihrem Protokoll und beginnen
% Sie hier mit der Lösung der ersten Aufgabe:



\begin{aufgabe}{Versuchsaufbau und Versuchsdurchführung}
  Beschreibung des Versuchsaufbaus einschließlich
  Schaltbild. Beschreibung der Versuchsdurchführung: verwendete
  Messwerterfassungseinstellungen, Messbereiche, Triggerbedingungen,
  etc.
\end{aufgabe}


\begin{aufgabe}{Vorversuch: Charakterisierung der verwendeten Bauteile}
  Charakterisieren Sie die verwendeten Bauteile mit Digitalvoltmeter
  bzw. Messbrücke.
\end{aufgabe}


\begin{aufgabe}{Ungekoppelte Schwingung}
  Zeigen Sie den Verlauf der Kondensatorspannungen für den
  ungekoppelten Fall und bestimmen Sie die Schwingungsfrequenz samt
  Messunsicherheit. Vergleichen Sie sie mit Ihrer Erwartung.
\end{aufgabe}


\begin{aufgabe}{Gekoppelte Schwingung: Schwebung}
  Stellen Sie für einen festen Kopplungsgrad $k$ eine Schwebung
  dar. Zeigen Sie jeweils die Fourierspektren, bestimmen Sie die
  Eigenfrequenzen und daraus den Kopplungsgrad sowie dessen
  Messunsicherheit. Bestimmen Sie die zeitliche Verschiebung
  $\Delta{}t$ zwischen den beiden Einhüllenden der Schwebungen der
  beiden Kondensatoren und vergleichen Sie sie mit Ihrer
  Erwartung. Stellen Sie dar, wie sich das Frequenzspektrum mit dem
  Abstand der Spulen ändert. Untersuchen Sie die Verstärkung der
  Kopplung durch Verwendung eines Eisenkerns in den beiden Spulen.
\end{aufgabe}


\begin{aufgabe}{Gekoppelte Schwingung: gleich- und gegensinnige Anregung}
  Stellen Sie für einen festen Kopplungsgrad $k$ nacheinander die
  beiden Fundamentalschwingungen dar. Zeigen Sie jeweils die
  Fourierspektren und bestimmen Sie die zugängliche
  Eigenfrequenz. Berechnen Sie den Kopplungsgrad sowie dessen
  Messunsicherheit. Vergleichen Sie mit den Werten aus der Schwebung.
\end{aufgabe}

 
\end{document}
