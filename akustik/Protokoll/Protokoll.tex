
\documentclass[twoside]{protokoll}
\usepackage{graphicx}
\praktikum{I}

\versuchsgebiet{(Akustik)}


\teilnehmer{Maximilian Carlos Menke, 434170}
\teilnehmer{Andrea Roth, 428396}
\gruppe{A3}

\begin{document}

\begin{versuchsziele}
Ziel des Versuches ist, das Elastizitätzsmodul verschiedener Metallstäbe zu bestimmen.
Eine statische Messung liefert nur für dünne Drähte Ergebnisse weswegen das Elastizitätsmodul in unserem Fall mit einer Dynamischen Messung bestimmt wird. Hierfür erzeugen wir stehende Wellen in den Metallstäben. So können wir mit der Frequenz und der Wellenlänge die Phasengeschwindigkeit und somit auch das Elastizitätsmodul bestimmen. 
\end{versuchsziele}

 
\section{1A3 Bestimmung des E-Moduls von Metallen}

\begin{aufgabe}{Grundlagen}
  % Knappe Beschreibung der theoretischen Grundlagen, Angabe der
  % Fbenötigten Formel(n), ohne Herleitung. Definition der verwendeten
  % Formelzeichen.
    Da wir keine Statische Messung des E-Moduls durchführen können, verwenden wir hier eine dynamische Messung.
    Für diese brauchen wir Grundlagen aus der Akustik über stehende Wellen, als auch von Festkoerpern.


    Das E-Modul ist ein Maß für die Dehnbarkeit eines Materials. Es ist Definiert als: 
    \begin{equation}
        E = \frac{\frac{F}{A}}{\frac{\Delta L}{L}}
    \end{equation}
    Es beschreibt wie sehr sich ein Material aus dehnt / komprimiert wenn ein Druck auf ihn ausgeübt wird.
    Es gilt allerdings nur im elastischen Bereich des Materials.\\


    Wenn der Stab in Schwingung versetzt wird, bliden sich stehende Wellen in ihm aus.
    Wir wollen hierbei die Frequenz der Grundschwingung bestimmen.
    Da der Stab an beiden Enden ein festes Ende für die Wellen hat, muss für Resonanz der Grundschwingung die Wellenlaenge: $\lambda = 2 * L$. 
     
    Daraus ergibt sich:
    \begin{equation}
        v = f * \lambda = f * 2 * L
    \end{equation}

    In Luft breitet sich Schall immer als Longitudinale Welle aus.
    In Festkörpern muss die Rückstellkraft des Materials nicht zwangsweise entgegen der Ausbreitungsrichtung zeigen, weshabl hier allgemein auch eine Transversalve Komponenten vorliegen kann.
    Die Metallstäbe können aber als homogen genung angenommen werden, weshalb wir hier von einer longitudinal Welle ausgehen können.
    Aus der Wellengleichung ergibt sich:
    \begin{equation}
         v = \sqrt{\frac{E}{p}}
    \end{equation}
    \begin{equation}
         E = v ^2 * p
    \end{equation}
    \begin{equation}
         E = f * \lambda = 4 * f_0 * L ^2 * p
    \end{equation}
    \begin{equation}
        E = f * \lambda = \frac{ 4 * f_0^2 * L ^2 * M}{\pi * (\frac{d}{2}) ^2}
    \end{equation}
    \begin{equation}
        E = f * \lambda = \frac{ 16 * f_0^2 * L ^2 * M}{\pi * d ^2}
    \end{equation}

     
\end{aufgabe}


\begin{aufgabe}{Versuchsaufbau und Versuchsdurchführung}
\subsection{Aufbau}
  Für die Bestimmung des E-Moduls messen wir die Schallwellen die mithilfe eines Gummihammers in dem Stab erzeugt wurden.
    Diese werden in ein Digitales Signal umgewandelt das wir mithilfe des Sensor CASSY darstellen können und analysieren.
    Hierzu benötigen wirfolgende Geräte.\\

    Für die Bestimmung des E-Moduls messen wir die Schallwellen die mithilfe eines Gummihammers in dem Stab erzeugt wurden.
    Diese werden in ein Digitales Signal umgewandelt das wir mithilfe des Sensor CASSY darstellen und analysieren können.
    Hierzu benötigen wir folgende Geräte.\\

\textbf{Benötigte Geräte:}
\begin{itemize}
\item Semsor CASSY
\item Universalmikrofon mit Stativstange
\item Sockel
\item Tischklemme
\item Metallstange (Länge: 20cm) (Stativstange für Kreuzmuffe)
\item Kreuzmuffe
\item Metallstift (Qürschnitt: 4mm, Länge: 30mm)
\item Gummi-Hammer
\item Mikrometerschraube (Messbereich: 0-25mm, Genauigkeit: $\pm$ 0.01mm)
\item Stahl-Bandmaß (Länge: 2m, Tolleranz: $\pm$ 0.7mm)
\item verschiedene Metallstangen (Kupfer, Messing, Stahl, Aluminium)
\item Analysewaage Sartorius BL 1500 (Genauigkeit: $\pm$ 0,2g)
\end{itemize}
\graphicspath{ {./} }
\includegraphics[width=15cm,height=15cm,keepaspectratio]{SkizzeAufbau}\\

Der Aufbau des Experiments kann der oben stehenden Skizze entnommen werden. 

Zunächst werden mit zwei Tischklemmen Mikrofon und Metallstange an dem Tisch befestigt. Wobei die Stange mithilfe der Kreuzmuffe befestigt wird. In die Kreuzmuffe wird der Metallstift ortogonal zu der Stange hineingelegt, sodass diese beim mittigen einspannen nur an einem Punkt unterstütz wird. Die Stange also frei schwingen kann. 

Das Universalmikrofon und die Metallstange werden auf eine Linie gebracht, mit 5mm Abstand. Sodass Das Mikrofon die Schallwellen gut aufzeichnen kann, aber nicht beschädigt wird, falls die Stange zu stark angeschlagen wird. Wärend der gesamten durchführung muss darauf geachtet werden, dass die Position der beiden sich nicht gegeneinander verschieben. 


Das Universalmikrofon wird in den Sensor CASSY eingesteckt, wobei das Gelbe Kabel(in der Skizze rot) in die Buxe für die Spannung, und das schwarze in die für die Erdung gesteckt wird. Das Mikrofon wird auf den Amplitudenmodus $\sim$ gestellt. Der Gummihammer wird verwendet um auf das, dem Mikrofon abgewandte, Ende der Metallstange zu schlagen. Damit werden die Metallstangen zum schwingen angeregt.

\subsection{Durchführung}

Zunächst haben wir die Metallstangen Vermessen. Dafür haben wir 
  
%  Beschreibung des Versuchsaufbaus einschließlich beschrifteter Skizze
 % oder Foto. Beschreibung der Versuchsdurchführung: Handgriffe an der
  %Apparatur, verwendete Messwerterfassungseinstellungen, Messbereiche,
 % Triggerbedingungen, etc.
\end{aufgabe}

\begin{aufgabe}{Rohdaten}
  Stellen Sie die gemessenen Werte von $L$, $D$ und $M$ in
  tabellarischer Form dar. Visualisieren Sie einige typische
  Schwingungsverläufe sowie deren Fourierspektren in geeigneter
  Weise.
\end{aufgabe}

\begin{aufgabe}{Auswertung}
  Bestimmen Sie aus den aufgezeichneten Schwingungsvorgängen die
  Schwingungsfreqünzen und tabellieren Sie die erhaltenen
  Werte. Ermitteln Sie die Streuung der einzelnen $f_i$ und bestimmen
  Sie daraus den statistischen Fehler auf den Mittelwert. Erläutern
  und illustrieren Sie Ihr Vorgehen zur Bestimmung der systematischen
  Unsicherheiten auf die Frequenz. Diskutieren Sie die Unsicherheiten
  auf $L$, $D$ und $M$. Berechnen Sie den E-Modul und die Dichte der
  vorliegenden Stangen und die zugehörigen statistischen und
  systematischen Messunsicherheiten. Diskutieren Sie, welche
  Fehlerbeiträge den Gesamtfehler dominieren. Vergleichen Sie Ihre
  Ergebnisse mit einschlägigen Literaturwerten.
\end{aufgabe}
 
\end{document}
