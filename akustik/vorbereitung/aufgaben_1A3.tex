\section{1A3 Bestimmung des E-Moduls von Metallen}

\begin{aufgabe}{Grundlagen}
  % Knappe Beschreibung der theoretischen Grundlagen, Angabe der
  % Fbenötigten Formel(n), ohne Herleitung. Definition der verwendeten
  % Formelzeichen.
    Da wir keine Statische Messung des E-Moduls durchführen können, verwenden wir hier eine dynamische Messung.
    Fuer diese brauchen wir sowohl die Grundlagen aus der Akustik ueber stehende Wellen, als auch die Grundlagen von Festkoerpern.

     
    Wenn der Stab in Schwingung versetzt wird, bliden sich stehende Wellen aus.
    Da der Stab an beiden Enden ein festes Ende fuer die Wellen in ihm hat, muss fuer Resonanz die Wellenlaenge: $\lambda = 2 * L$. 
     
    Dies kann umgeformt werden zu:
    \begin{equation}
        v = f * \lambda = f * 2 * L
    \end{equation}

    Aus der Lösung der Wellengleichung ergibt sich:
    \begin{equation}
         v = \sqrt{\frac{E}{p}}
    \end{equation}
    \begin{equation}
         E = v ^2 * p
    \end{equation}
    \begin{equation}
         E = f * \lambda = f * 2 * L ^2 * p
    \end{equation}

     
\end{aufgabe}

% Bitte belassen Sie die Aufgabentexte in Ihrem Protokoll und beginnen
% Sie hier mit der Lösung der ersten Aufgabe:



\begin{aufgabe}{Versuchsaufbau und Versuchsdurchführung}
  Für die Bestimmung des E-Moduls messen wir die Schallwellen die mithilfe eines Gummihammers in dem Stab erzeugt wurden. Diese werden in ein Digitales Signal umgewandelt das wir mithilfe des Sensor CASSY darstellen können und analysieren. Hierzu benötigen wir:\\

\textbf{Benötigte Geräte:}
\begin{itemize}
\item Semsor CASSY
\end{itemize}
  
  
  
  Beschreibung des Versuchsaufbaus einschließlich beschrifteter Skizze
  oder Foto. Beschreibung der Versuchsdurchführung: Handgriffe an der
  Apparatur, verwendete Messwerterfassungseinstellungen, Messbereiche,
  Triggerbedingungen, etc.
\end{aufgabe}


\begin{aufgabe}{Rohdaten}
  Stellen Sie die gemessenen Werte von $L$, $D$ und $M$ in
  tabellarischer Form dar. Visualisieren Sie einige typische
  Schwingungsverläufe sowie deren Fourierspektren in geeigneter
  Weise.
\end{aufgabe}


\begin{aufgabe}{Auswertung}
  Bestimmen Sie aus den aufgezeichneten Schwingungsvorgängen die
  Schwingungsfrequenzen und tabellieren Sie die erhaltenen
  Werte. Ermitteln Sie die Streuung der einzelnen $f_i$ und bestimmen
  Sie daraus den statistischen Fehler auf den Mittelwert. Erläutern
  und illustrieren Sie Ihr Vorgehen zur Bestimmung der systematischen
  Unsicherheiten auf die Frequenz. Diskutieren Sie die Unsicherheiten
  auf $L$, $D$ und $M$. Berechnen Sie den E-Modul und die Dichte der
  vorliegenden Stangen und die zugehörigen statistischen und
  systematischen Messunsicherheiten. Diskutieren Sie, welche
  Fehlerbeiträge den Gesamtfehler dominieren. Vergleichen Sie Ihre
  Ergebnisse mit einschlägigen Literaturwerten.
\end{aufgabe}



