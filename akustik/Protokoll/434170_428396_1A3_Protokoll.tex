\documentclass[twoside]{protokoll}
\usepackage{graphicx}
\usepackage{tabularx} % for better table formatting
\usepackage{booktabs} % for better table formatting
\usepackage{float} % for preservint the order of figures and tables
\praktikum{I}
\usepackage{subfig}
\usepackage{amsmath}

\versuchsgebiet{(Akustik)}


\teilnehmer{Maximilian Carlos Menke, 434170}
\teilnehmer{Andrea Roth, 428396}
\gruppe{A3}

\begin{document}

\begin{versuchsziele}
Ziel des Versuches ist, das Elastizitätsmodule verschiedener Metallstäbe zu bestimmen.
Eine statische Messung liefert nur für dünne Drähte Ergebnisse weswegen das Elastizitätsmodule in unserem Fall mit einer Dynamischen Messung bestimmt wird. Hierfür erzeugen wir stehende Wellen in den Metallstäben. So können wir mit der Frequenz und der Wellenlänge die Phasengeschwindigkeit und somit auch das Elastizitätsmodule bestimmen. 
\end{versuchsziele}

 
\section{1A3 Bestimmung des E-Moduls von Metallen}


\begin{aufgabe}{Grundlagen}
  % Knappe Beschreibung der theoretischen Grundlagen, Angabe der
  % Fbenötigten Formel(n), ohne Herleitung. Definition der verwendeten
  % Formelzeichen.
    Da wir keine Statische Messung des E-Moduls durchführen können, verwenden wir hier eine dynamische Messung.
    Für diese brauchen wir Grundlagen aus der Akustik über stehende Wellen, als auch von Festkörpern.


    Das E-Modul ist ein Maß für die Dehnbarkeit eines Materials. Es ist Definiert als: 
    \begin{equation}
        E = \frac{\frac{F}{A}}{\frac{\Delta L}{L}}
    \end{equation}
    Es beschreibt wie sehr sich ein Material aus dehnt / komprimiert wenn ein Druck auf ihn ausgeübt wird.
    Es gilt allerdings nur im elastischen Bereich des Materials.\\
    
    Wenn wir mit dem Hammer auf das Ende des Stabes schlagen, so entsteht eine Druckwelle in ihm.
    Diese durchläuft den Stab. Am andren Ende des wird dieser ein kleines Stück ausgedehnt.
    Dieses sorgt dafür, das die Druckänderung an die Luft übertragen wird.
    Da der Stab aber ein Elastizitätsmodule besitzt, gibt es eine Rückstellkraft die die Ausdehnung des Stabes wieder zurückführt.
    Diese Druckschwankung durchläuft den Stab periodisch.
    Dadurch entsteht eine Stehende Welle in dem Stab, welche über die Enden des Stabes Schallwellen an die Umgebung abgibt.

    Wenn der Stab in Schwingung versetzt wird, bilden sich stehende Wellen in ihm aus.
    Wir wollen hierbei die Frequenz der Grundschwingung bestimmen.
    Da der Stab an beiden Enden ein festes Ende für die Wellen hat, muss für Resonanz der Grundschwingung die Wellenlänge: $\lambda = 2 L$. 
     
    Daraus ergibt sich: \\
    
    \begin{equation}
        v = f \lambda = f 2 L
    \end{equation}\\

    In Luft breitet sich Schall immer als Longitudinale Welle aus.
    In Festkörpern muss die Rückstellkraft des Materials nicht zwangsweise entgegen der Ausbreitungsrichtung zeigen, weshalb hier allgemein auch eine Transversale Komponenten vorliegen kann.
    Die Metallstäbe können aber als homogen genug angenommen werden, weshalb wir hier von einer longitudinal Welle ausgehen können.
    Aus der Wellengleichung ergibt sich:
    \begin{equation}
         v = \sqrt{\frac{E}{\rho}}
    \end{equation}
Hierbei bezeichnet $\rho$ die Dichte des Materials, v die Phasengeschwindigkeit der 
longitutionalen Welle und E das E-Modul. Die dichte ist gegeben als:
	\begin{equation}
	\rho = \frac{M}{V} \quad  mit \quad V = \pi\left(\frac{D}{2}\right)^2*L
	\end{equation}
    \begin{equation}
         E = v ^2 \rho
    \end{equation}
    \begin{equation}
        E = (f \lambda)^2 \rho = 4 f_0 L ^2 \rho
    \end{equation}
    \begin{equation}
        E = \frac{ 16 f_0^2 L M}{\pi D ^2}
    \end{equation}

     
\end{aufgabe}

\begin{aufgabe}{Versuchsaufbau und Versuchsdurchführung}
\subsection{Versuchsaufbau}
  Für die Bestimmung des E-Moduls messen wir die Schallwellen die mithilfe eines Gummihammers in dem Stab erzeugt wurden.
    Diese werden in ein Digitales Signal umgewandelt das wir mithilfe des Sensor CASSY darstellen können und analysieren.
    Hierzu benötigen wir folgende Geräte.\\

    Für die Bestimmung des E-Moduls messen wir die Schallwellen die mithilfe eines Gummihammers in dem Stab erzeugt wurden.
    Diese werden in ein Digitales Signal umgewandelt das wir mithilfe des Sensor CASSY darstellen und analysieren können.
    Hierzu benötigen wir folgende Geräte.\\

\textbf{Benötigte Geräte:}
\begin{itemize}
\item Sensor CASSY
\item Universalmikrofon mit Stativstange
\item Sockel
\item Tisch klemme
\item Metallstange (Länge: 20cm) (Stativstange für Kreuzmuffe)
\item Kreuzmuffe
\item Metallstift (Querschnitt: 4mm, Länge: 30mm)
\item Gummi-Hammer
\item Mikrometer schraube (Messbereich: 0-25mm, Genauigkeit: $\pm$ 0.01mm)
\item Stahl-Band-maß (Länge: 2m, Toleranz: $\pm$ 0.7mm)
\item verschiedene Metallstangen (Kupfer, Messing, Stahl, Aluminium)
\item Analysewaage Sartorius BL 1500 (Genauigkeit: $\pm$ 0,2g)
\end{itemize}


\begin{figure}[H]
  \centering
  \includegraphics[width=1\textwidth]{Bilder/434170_428396_1A3_SkizzeAufbau.pdf}
  \caption{Skizze des Versuchsaufbaus}
  \centering
\end{figure}
 
Der Aufbau des Experiments kann der oben stehenden Skizze entnommen werden. \\


Zunächst werden mit zwei Tisch klemmen Mikrofon und Metallstange an dem Tisch befestigt. Wobei die Stange mithilfe der Kreuzmuffe befestigt wird. In die Kreuzmuffe wird der Metallstift orthogonal zu der Stange hineingelegt, sodass diese beim mittigen einspannen nur an einem Punkt unterstützt wird. Die Stange also frei schwingen kann. 

Das Universalmikrofon und die Metallstange werden auf eine Linie gebracht, mit 5mm Abstand. Sodass Das Mikrofon die Schallwellen gut aufzeichnen kann, aber nicht beschädigt wird, falls die Stange zu stark angeschlagen wird. Während der gesamten Durchführung muss darauf geachtet werden, dass die Position der beiden sich nicht gegeneinander verschieben. 


Das Universalmikrofon wird in den Sensor CASSY eingesteckt, wobei das Gelbe Kabel(in der Skizze rot) in die Buxe für die Spannung, und das schwarze in die für die Erdung gesteckt wird. Das Mikrofon wird auf den Amplitudenmodus $\sim$ gestellt. Der Gummihammer wird verwendet um auf das, dem Mikrofon abgewandte, Ende der Metallstange zu schlagen. Damit werden die Metallstangen zum schwingen angeregt.

\subsection{Durchführung}

Zunächst haben wir die Metallstangen kategorisiert nach dem Material aus welchem sie bestehen.\\

\textbf{Kathegorisierung der Metallstangen:}

\begin{itemize}
\item Aluminium:		 matt silberne Stange 
\item Messing:		 goldene Stange
\item Kupfer:			 kupferfarbene Stange
\item Stahl 15:		 Silber glänzende Stange
\end{itemize}

Von diesen haben wir dann jeweils, Länge, Masse und Durchmesser bestimmt. 
Die Länge haben wir mit dem Band-maß gemessen, wobei hier darauf geachtet wurde, dass das Band-maß straff ist. Die Masse haben wir mit der Analyse wage bestimmt, indem wir die Stange mittig auf der Wage platziert haben. 


Da wir bei der Stange annehmen, dass diese einen kreisförmigen Querschnitt hat, was nicht perfekt zutreffen wird, führen wir die Messung des Durchmessers mehrfach durch. Hierfür verwenden wir die Mikrometer schraube. Wir messen an verschiedenen Stellen, dabei rotieren wir die Stange beliebig bei jeder Messung um die längsachse. Dies wiederholen wir zehn mal. Sodass wir für jede Stange 10 Messwerte für den Durchmesser haben.\\ 

\begin{figure}[H]
  \centering
  \includegraphics[width=1.0\textwidth]{Bilder/434170_428396_1A3_Materialien.pdf}
  \caption{Bild der Messgerät und Metallstangen}
  \centering
\end{figure}

Als nächstes bauen wir den Versuch auf. Hier gehen wir genau so vor wie in der vorherigen Beschreibung des Versuchsaufbaus. Als erstes haben wir die Kupferstange verwendet.
Wobei wir zunächst die Tisch klemme platziert haben und die Stange mittig eingespannt. Hierfür haben wir von einem Ende der Stange aus, mit dem Maß-band die Mitte bestimmt. Dann haben wir das Mikrofon passend zu der Stange positioniert, und an den Tisch montiert. Hierbei haben wir noch beachtet, dass die Stange nicht zu fest eingespannt werden sollte, da dies Auswirkung auf die Frequenz hat. Außerdem haben wir überprüft, dass sie nur an einem Punkt aufliegt, sodass sie frei schwingen kann. Die Spitze des Mikrofons haben wir ungefähr 5mm entfernt vom Ende der Stange positioniert. Dies haben wir nach Augenmaß getan. 

\begin{figure}[H]
  \centering
  \includegraphics[width=1\textwidth]{Bilder/434170_428396_1A3_Gesamtaufbau.pdf}
  \caption{Bild der fertig aufgebauten Versuchsanordnung}
  \centering
\end{figure}

Dann haben wir das Mikrofon wie im Aufbau beschrieben an das Sensor CASSY angeschlossen und in betrieb genommen. Zunächst haben wir uns überlegt, was sinnvolle Messparameter sind. Da die Schallgeschwindigkeit in Metallen im Bereich von mehrere 1000m/s liegt, muss das Messintervall im Bereich von 10-100$\mu$s sein. Somit haben wir erste Testmessung durchgeführt. Wobei wir diese mehrfach wiederholt haben, bis wir die Empfindlichkeit des Mikrofons so eingestellt hatten, dass wir den gesamten dynamischen Bereich des Mikrofons ausnutzen, und nicht in die Sättigung kommen. Dann konnten wir eine erste FFT mit CASSY durchführen, und mit peakfinder die Frequenz des schwingenden Kupferstabs ungefähr bestimmen. Mit diesen Informationen haben wir dann unsere Messparameter eingestellt.\\

\textbf{Messparameter:}
\begin{itemize}
\item Messzeit: 3s
\item Intervall: 100$\mu$s
\item Anzahl Messungen: 30001
\item Spannungsbereich: -3V bis 3V (Da der dynamisch Bereich des Mikrofons 2.5V ist)
\end{itemize}

Wir haben für die Messung keinen Trigger verwendet, sondern die Messung manuell gestartet. 


Bei einer Messung hat einer aus unserer Gruppe mit dem Gummihammer den Kupferstab angeschlagen, während die andere Person kurz nach Anschlag die Messung gestartet hat, so dass, der Einschwingvorgang möglichst nicht mit aufgezeichnet wurde. Wir haben versucht den Stab immer möglichst gleich an zu schlagen, also gleiche Position und Kraft, damit die Messungen möglichst vergleichbar sind. Außerdem haben wir darauf geachtet, dass wir die Stange nur dann anschlagen, wenn niemand anders eine Stange des gleichen Materials anschlug.

\begin{figure}[H]
  \centering
  \subfloat[Anschlagen der Stange]{\includegraphics[width=0.3\textwidth]{Bilder/434170_428396_1A3_Hammer.pdf}\label{fig:f1}}
  \hfill
  \subfloat[Position von Stange und Mikrofon]{\includegraphics[width=0.5\textwidth]{Bilder/434170_428396_1A3_MSNahaufnahme.pdf}\label{fig:f2}}
  \caption{Anordnung zur Durchführung einer Messung}
\end{figure}



Mit unseren eingestellten Messparametern haben wir dann eine erste vollständige Testmessung durchgeführt um zu überprüfen, ob alle Einstellungen gut passen. Aus dieser Messung haben wir erneut mit der FFT die Frequenz der Grundschwingung bestimmt. Mit dieser konnten wir eine erste Überschlagsrechnung für das E-Modul von Kupfer anstellen.
Diese ist im Messprotokoll zu finden. Da der Wert in der erwarteten Größenordnung lag, haben wir die Messreihe mit diesem Aufbau gestartet. 


Insgesamt haben wir die Messung gleich wie bei der Testmessung 10 mal durchgeführt, und die Ergebnisse mithilfe der FFT des CASSY Lab 2 grob auf Konsistenz überprüft. Diese Messungen haben wir als unsere Messreihe abgespeichert.\\

Die Frequenz ist auch abhängig von der Einspannung der Stange, also von der Kraft und der Position und der Orientierung der Einspannung. Weswegen der Fehler aufgrund dieser Effekte auch beachtet werden muss. Dazu haben wir mit der Kupferstange Messungen durchgeführt bei denen wir die Einspannung um 1cm-2cm um den Mittelpunkt variiert haben, sowie die Kraft mit der sie eingespannt ist, als auch die Rotation um die Längsachse.
Diese haben wir als Messreihe für spätere Bestimmung des Fehlers gespeichert.\\

\begin{figure}[H]
  \centering
  \includegraphics[width=0.8\textwidth]{Bilder/434170_428396_1A3_Einspannung2.pdf}
  \caption{Einspannungspunkt der Stange}
  \centering
\end{figure}

Das selbe Vorgehen wie bei der Kupferstange haben wir auch bei den anderen Stangen durchgeführt. Wobei wir jedes mal zunächst eine Testmessung gemacht habe, um die Empfindlichkeit des Mikrofons für das Material spezifisch ein zu stellen, und zu sehen ob die gemessene Frequenz, und damit auch das E-Modul den Erwartungen entspricht. 

Sämtliche gemessenen Daten haben wir gespeichert, sodass wir diese zur Auswertung verwenden können. einige Messungen mussten wir wiederholen, da wir die Stange nicht richtig angeschlagen hatten, oder sie sich beim Anschlag vom Mikrofon weg gedreht hat. Ach durch Störgeräusche anderer Stangen, oder lautes Reden wurden einige Messungen gestört weswegen wir sie erneut durchführten.\\

Nach Beendigung aller Messungen haben wir alles wieder abgebaut und an die dafür vorgesehenen Orte zurückgelegt. 


\end{aufgabe}

\begin{aufgabe}{Rohdaten}

Der unten stehenden Tabelle können die Masse und die Länge der vermessenen Metallstangen entnehmen. \\
    \begin{table}[H]
        \centering
        \begin{tabularx}{0.8\textwidth}{X c c} % adjust width as needed
            \toprule
            \textbf{Metall Stange} & \textbf{Masse} & \textbf{Länge} \\
            \midrule
            Aluminium & 460.9g & 150cm \\
            Messing & 1427.5g & 150cm \\
            Kupfer & 1505.4g & 150cm \\
            Stahl 15 & 1327.4g & 150cm \\
            \bottomrule
        \end{tabularx}
        \caption{Masse und Länge der Stäbe}
        \label{tab:mytable}
    \end{table}

In dieser Tabelle finden sie die Messung des Durchmessers. Diese haben wir pro Stange 10 mal durchgeführt. Gemessen wurde mit der Milimeterschraube an verschiedenen Positionen.\\
    \begin{table}[H]
        \centering
        \begin{tabularx}{0.8\textwidth}{X c c c c} % adjust width as needed
            \toprule
            \textbf{Messung} & \textbf{Aluminium} & \textbf{Messing} & \textbf{Kupfer} & \textbf{Stahl 15} \\
            \midrule
            1. & 12.05mm & 11.98mm & 11.98mm & 12.00mm \\
            2. & 12.05mm & 12.01mm & 11.98mm & 12.00mm \\
            3. & 12.06mm & 11.98mm & 11.98mm & 11.99mm \\
            4. & 12.05mm & 11.99mm & 11.98mm & 12.00mm \\
            5. & 12.06mm & 11.98mm & 11.98mm & 12.00mm \\
            6. & 12.06mm & 11.98mm & 11.98mm & 12.00mm \\
            7. & 12.06mm & 11.99mm & 11.98mm & 12.00mm \\
            8. & 12.05mm & 11.99mm & 11.98mm & 12.00mm \\
            9. & 12.06mm & 11.98mm & 11.98mm & 12.00mm \\
            10.& 12.07mm & 11.98mm & 11.98mm & 12.01mm \\
            \bottomrule
        \end{tabularx}
        \caption{10mal Messung der Durchmesser der Stäbe}
        \label{tab:mytable}
    \end{table}

Im Verlauf der Durchführung des Versuches haben wir verschiedene Messreihen aufgenommen. Wobei wir für jeden Stab eine Reihe von 10 Messungen durchgeführt haben. Bei dem Kupferstab haben wir noch eine Messreihe aufgenommen, die wir später verwenden werden um den Systematischen Fehler aufgrund der Einspannung zu bestimmen. Zur Übersichtlichkeit und damit im Späteren Verlauf klar ist Welche Messreihe welche ist, hier noch eine Auflistung von diesen:\\

    \begin{table}[H]
        \centering
        \begin{tabularx}{1\textwidth}{c X X} 
            \toprule
            \textbf{Messreihe} & \textbf{Name} & \textbf{Beschreibung} \\
            \midrule
            1. & Kupfer\_Messung & 10mal Messung der Schwingung der Kupferstange \\\\
            2. & Alu\_Messung & 10mal Messung der Schwingung der Aluminiumstange \\\\
            3. & Stahl\_Messung & 10mal Messung der Schwingung der Stahlstange \\\\
            4. & Messing\_Messung & 10mal Messung der Schwingung der Messingstange \\\\
            5. & Kupfer\_Einsp\_Fehler & 6mal Messung der Schwingung bei verschiedenen 				Einspannpositionen Orientierungen und Kraft.  \\\\  
            6. & Material\_Test & Pro Material einmalige Testmessung, zur Überprüfung das alle Parameter stimmen \\
            \bottomrule
        \end{tabularx}
        \caption{Übersicht der Messreihen}
        \label{tab:mytable}
    \end{table}

     
    Alle folgenden Plots wurden mit dem Programm programme/show\_all\_plots.py erstellt.
    Die FFT haben wir in den Plots bis 5000Hz dargestellt, da wir mit einer Auflösung von 100$\mu s$ gemessen haben und nach dem Nequist Theorem theoretisch in der Fourieranalyse Frequenzen bis 5000Hz auflöst werden können.
\begin{figure}[H]
    \caption{Exemplarische Messung für jedes Material und zugehörige Fouierspektren}
  \centering
  \subfloat{\includegraphics[width=0.5\textwidth]{plots/434170_428396_1A3_Messing_Messung_01_plot.pdf}}
  \subfloat{\includegraphics[width=0.5\textwidth]{plots/434170_428396_1A3_Alu_Messung_01_plot.pdf}}
\end{figure}
\begin{figure}[H]
  \centering
  \subfloat{\includegraphics[width=0.5\textwidth]{plots/434170_428396_1A3_Messing_Messung_01_fft.pdf}}
  \subfloat{\includegraphics[width=0.5\textwidth]{plots/434170_428396_1A3_Alu_Messung_01_fft.pdf}}
\end{figure}
\begin{figure}[H]
  \centering
  \subfloat{\includegraphics[width=0.5\textwidth]{plots/434170_428396_1A3_Stahl_Messung_02_plot.pdf}}
  \subfloat{\includegraphics[width=0.5\textwidth]{plots/434170_428396_1A3_Kupfer_Messung_02_plot.pdf}}
\end{figure}
\begin{figure}[H]
  \centering
  \subfloat{\includegraphics[width=0.5\textwidth]{plots/434170_428396_1A3_Stahl_Messung_02_fft.pdf}}
  \subfloat{\includegraphics[width=0.5\textwidth]{plots/434170_428396_1A3_Kupfer_Messung_02_fft.pdf}}
\end{figure}

\begin{figure}[H]
    \caption{Exemplarische Messung mit Störungen und zugehörigen Fouierspektren}
  \centering
    \subfloat[Sättigung vom Mikrophon am Anfang der Messung]{\includegraphics[width=0.5\textwidth]{plots/434170_428396_1A3_Kupfer_Messung_09_plot.pdf}}
    \subfloat[Viele Störgeräusche]{\includegraphics[width=0.5\textwidth]{plots/434170_428396_1A3_Alu_Messung_09_plot.pdf}}
\end{figure}
\begin{figure}[H]
  \centering
    \subfloat{\includegraphics[width=0.5\textwidth]{plots/434170_428396_1A3_Kupfer_Messung_09_fft.pdf}}
    \subfloat{\includegraphics[width=0.5\textwidth]{plots/434170_428396_1A3_Alu_Messung_09_fft.pdf}}
\end{figure}

Sämtlichen Rohdaten befinden sich im Ordner Messungen. Die Namen der Datei haben die 
Struktur 434170\_428396\_Name\_Messreihe.
Die Bilder befinden sich im Ordner Bilder als PDF abgespeichert.
\end{aufgabe}

\begin{aufgabe}{Auswertung}

\subsection{Aufarbeitung Rohdaten}

    
    Zunächst haben wir unsere gemessenen Daten der Schwingung aufbereitet und selektiert.
    Dies ist sinnvoll, da es verschiedene Fehlerquellen gibt (Störgeräusche, unsaubere Messung, Übersteuerung, etc.) die das Ergebnisse verfälschen können.
    Diese Fehlerquellen können einen signifikanten Einfluss auf das Endgültige Ergebnis haben.
    Um diesen Einfluss möglichst gering zu halten, sollten deshalb die Rohdaten alle angeschaut und aufbereitet werden.
    Hierbei muss auf verschiedene Dinge geachtet werden.
    Diese werden im Folgenden an Beispielen Diskutiert. 

\begin{figure}[H]
    \caption{Eine Messung mit vielen Störgeräusche, und trotzdem guter Fourieranalyse}
  \centering
    \subfloat{\includegraphics[width=0.5\textwidth]{plots/434170_428396_1A3_Alu_Messung_07_plot.pdf}}
    \subfloat{\includegraphics[width=0.5\textwidth]{plots/434170_428396_1A3_Alu_Messung_07_fft.pdf}}
\end{figure}
 
Oben können sie Exemplarischen den Graphen der Messung Alu\_ Messung \_ 07 ehen, bei der es viele Störgeräusche gab.
Wenn wir uns nun jedoch das Frequenzspektrum anschauen, können wir diese Störgeräusche klar von der Frequenz des Stabes unterscheiden.
Im Frequenzbereich des Stabes, sind keine störenden Geräusche, weswegen bei diesen Messdaten keine Daten punkte entfernt werden müssen zur Erhöhung der Genauheit.
Dies gilt auch für alle weiteren Messungen in welchen Störgeräusche vorhanden waren. 

Bei einigen Messungen wurde der Stab anfangs zu stark angeschlagen, weswegen die Gemessene Amplitude anfangs im gesättigten Bereich liegt.
Da dies die gemessene Grenzfrequenz beeinflussen kann  haben wir bei diesen Messungen die Daten bis zu dem Punkt, wo die Amplitude unter 2.5V ist, weg geschnitten.
Dies verfälscht das Ergebnis nicht, da es nur ein kleines Intervall ist, und wir somit trotzdem Daten punkte von mehr als 2.5s Messung haben.
Dies ist ein ausreichend großes Intervall.\\
In der Durchführung wurde bereits diskutiert, dass wir Messintervalle von 100$\mu$s verwenden können und damit eine ausreichende Abtastrate haben.
Somit werden auch in einem Intervall von 2.5s  genügend Messpunkte aufgenommen, um daraus die Frequenz ermitteln zu können.

\begin{figure}[H]
  \centering
  \caption{Messung ohne weggeschnittenem Anfang}
  \subfloat{\includegraphics[width=0.5\textwidth]{plots/434170_428396_1A3__Kupfer_Einsp_Fehler_01_plot.pdf}}
  \hfill
  \subfloat{\includegraphics[width=0.5\textwidth]{plots/434170_428396_1A3__Kupfer_Einsp_Fehler_01_fft.pdf}}
\end{figure}

\begin{figure}[H]
  \centering
  \caption{Messung mit weggeschnittenem Anfang}
  \subfloat{\includegraphics[width=0.5\textwidth]{plots/434170_428396_1A3_Kupfer_Einsp_Fehler_01_plot.pdf}}
  \hfill
  \subfloat{\includegraphics[width=0.5\textwidth]{plots/434170_428396_1A3_Kupfer_Einsp_Fehler_01_fft.pdf}}
\end{figure}

Oben können sie einen Exemplarischen Graphen von unserer Messung: Kupfer\_Einsp\_Fehler\_01 sehen, in welchem man sehr gut sehen kann, dass anfangs die Amplitude im gesättigten Bereich war, und somit auch nichts über 2.5V aufgezeichnet wurde.
In dieser Messung haben wir also 0.5s weggeschnitten damit der mögliche Effekt davon keinen Einfluss auf unser Ergebnis haben.
Sie können auch die FFT dieser Messung sehen, und den daraus berechneten Peak.
Wobei wir bei der FFT nur dem Relevanten Bereich zwischen 0 und 2000 Hz dargestellt haben.
An dem Wert des Peaks können Sie sehen, dass das Wegschneiden des Anfangs einen Einfluss hatte.
Dieser tritt jedoch erst in der zweiten Nachkommastelle auf.
Ist also ein geringer Einfluss.
Wir wollen ihn jedoch der Genauheithalber nicht vernachlässigen.
Und schneiden somit bei den Messdaten bei denen anfangs überteuert wurde einen Teil ab. \\


Eine genaue Auflistung von allen Daten bei denen wir den Anfang oder das Ende weggeschnitten haben, zur Bestimmung der Frequenz wird unten angegeben.
Dieses hier ist nur eine Exemplarische Diskussion warum dies sinnvoll ist.

Wir hatten bei unserer Messreihe 5. eine Messung (Kupfer\_einsp\_Fehler\_05) bei welcher auf Grund von Störgeräusche und anderen Messeinflüssen wir ein sehr unsauberes Frequenzspektrum nach der Fourieranalyse erhalten haben.
\begin{figure}[H]
  \centering
    \caption{Messung mit sehr starken Störungen auch über 1000 Hz}
  \subfloat{\includegraphics[width=0.5\textwidth]{plots/434170_428396_1A3_Kupfer_Einsp_Fehler_05_plot.pdf}}
  \hfill
  \subfloat{\includegraphics[width=0.5\textwidth]{plots/434170_428396_1A3_Kupfer_Einsp_Fehler_05_fft.pdf}}
\end{figure}

In der FFT kann man sehr gut erkennen, dass wir auch eine vergleichsweise starke Streuung der Frequenzen im Bereich der zu Ermittelnden Grenzfrequenz haben.
Aufgrund der ungenauen Messung haben wir sie nicht bei der Auswertung berücksichtigt.
Diese ist aber die einzige Messung, welche wir nicht berücksichtigt haben. \\
 

\textbf{Liste der Messungen bei denen Teile weggeschnitten wurden:}
\begin{itemize}
\item Alu\_Messung\_8 bis 0.45s
\item Kupfer\_Einsp\_Fehler\_1 bis 0.5s
\item Kupfer\_Einsp\_Fehler\_5 ab 2s EVLT LOESCHEN
\item Kupfer\_Messung\_3 bis 0.5s
\item Kupfer\_Messung\_4 bis 0.5s
\item Kupfer\_Messung\_6 bis 0.5s
\item Kupfer\_Messung\_7 bis 0.5s
\item Kupfer\_Messung\_9 bis 0.5s
\item Messing\_Messung\_6 bis 0.5s
\item Messing\_Messung\_7 bis 0.5s
\item Stahl\_Messung\_1 bis 0.2s
\item Stahl\_Messung\_9 bis 0.5s
\end{itemize}
% TODO kupfer testmessung mit messung 1 austauschen

 
\subsection{Auswertung der Schwingungsmessungen}
Aus unseren Messreihen 1 bis 4 konten wir mithilfe einer FFT die wir mit Python durchgeführt haben die Grundfrequenz bestimmen.
Hierfür haben wir das Lokale Maximum im Intervall 1000 bis 2000 Herz ermittelt, um Umgebungsrauschen nicht zu berücksichtigen.
Da wir für Metalle erwarten, dass die Schwingungsfrequenz in diesem Bereich liegt, und eine Analyse der Rohdaten dieses auch für alle Messungen bestätigt hat, können wir dieses verlustfrei machen.
So konnte auch sichergestellt werden, dass wir nicht die Peaks von Störgeräusche (Reden, andere Stangen, etc.) ermitteln.
In der Folgenden Tabelle werden unsere Ergebnisse für die Frequenzen der einzelnen Stangen dargestellt.


 \begin{table}[H]
        \centering
        \begin{tabularx}{1.00\textwidth}{X X X X X} % adjust width as needed
            \toprule
            \textbf{Messung Nr.} & \textbf{Stahl 15} & \textbf{Aluminium} & \textbf{Messing} & \textbf{Kupfer} \\
            \midrule
                Nr. 1 & 1726.73Hz & 1673.88Hz & 1166.31Hz & 1288.93Hz \\
                Nr. 2 & 1726.02Hz & 1673.85Hz & 1166.53Hz & 1288.11Hz \\
                Nr. 3 & 1726.21Hz & 1673.61Hz & 1166.28Hz & 1288.03Hz \\
                Nr. 4 & 1726.45Hz & 1673.73Hz & 1166.32Hz & 1288.15Hz \\
                Nr. 5 & 1726.29Hz & 1673.67Hz & 1166.27Hz & 1287.99Hz \\
                Nr. 6 & 1726.41Hz & 1673.81Hz & 1166.47Hz & 1288.03Hz \\
                Nr. 7 & 1726.64Hz & 1673.02Hz & 1166.29Hz & 1288.02Hz \\
                Nr. 8 & 1726.76Hz & 1673.52Hz & 1166.32Hz & 1288.13Hz \\
                Nr. 9 & 1726.90Hz & 1673.90Hz & 1166.41Hz & 1288.13Hz \\
                Nr. 10& 1726.88Hz & 1673.60Hz & 1166.79Hz & 1288.66Hz \\
            \bottomrule
        \end{tabularx}
        \label{tab:mytable}
    \end{table}

\subsection{Bestimmung der Fehler und Fehlerfortplanzung}

Für die Auswertung der Daten und der Endgültigen Berechnung der Elastizitätsmodule, müssen wir die Fehler 
auf folgende Größen beachten. 
Hier sind sowohl die Statistischen Fehler als auch die Systematischen Fehler relevant.\\


\textbf{Fehlerbehaftete Größen:}
\begin{itemize}
\item Länge der Metallstangen (Systematischer Fehler)
\item Masse der Metallstangen (Systematischer Fehler)
\item Durchmesser der Metallstangen (Systematischer und Statistischer Fehler)
\item Gemessene Schwingungsfrequenz (Systematischer und Statistischer Fehler)
\end{itemize}

\subsubsection{Systematische Fehler}
Zunächst ein mal die Systematischen Fehler, diese müssen getrennt von den Statistischen Fehlern fortgepflanzt werden.
Wir haben einen Systematischen Fehler auf alle unsere Messgrößen.
Bei der Längenmessung rührt dies von der Genauigkeit des Band-maß, dies hängt von der Güteklasse des Band-maß ab.
In unserem Fall haben wir die Güteklasse II.
Bei der Analyse wage  hängt dieser auch von der Genauigkeit ab, so wie bei der Mikrometer schraube. 
 
Diese Informationen werden vom Hersteller gegeben, und können im Datenblatt des Geräts nachgeschlagen werden. 

Der Systematische Fehler bei der Frequenz kommt durch die Digitalisierung des akustischen Signals. Hierfür ist die verwendete Spannungsintervallbreite und die Auflösung (12Bit) relevant. Daraus berechnet sich der Systematische Fehler folgendermaßen.\\
 Wobei $ \pmb{\Delta_{dig}} $ der Systematische Fehler auf die Frequenz auf grund der digitalisierung ist, \textbf{I} die Spannungsintervallbreite (Da wir einen Bereich von -3V bis 3V gewählt haben) und \textbf{A} die Auflösung. Im Folgenden bezeichnen ein $\pmb{\Delta_i}$ immer einen Systematischen Fehler auf eine Messgröße und $ \pmb{\sigma_i}$ einen statistischen Fehler.\\ 
\begin{equation}
         a = \frac{I}{A}
    \end{equation}
\begin{equation}
         \Rightarrow 
         \Delta_{dig} = \frac{a}{\sqrt{12}}
\end{equation}

Zusätzlich zu diesem Systematischen Fehler, erhalten wir noch einen Systematischen Fehler durch den Einfluss der Einspannposition.
Dabei gehen wir davon aus, das der Systematische Fehler auf die Frequenz bei allen Stangen gleich ist.
Deshalb haben wir zur Fehler Abschätzung nur eine Messreihe mit der Kupfer Stange durchgeführt.
Dafür haben wir die Einspannposition um $\pm{0.5}{cm}$ verändert.
Dann haben wir noch einmal die Rotation der Stange um zufällige Werte verändert, und den Druck der Schraube (lose, mittel, fest) veränder.
 

\begin{table}[H]
    \centering
    \begin{tabularx}{0.7\textwidth}{X l} % adjust width as needed
        \toprule
        \textbf{Test Messung bei Kupfer} & \textbf{Frequenz} \\
        \midrule
            1 & 1288.18Hz \\
            2 & 1288.26 Hz \\
            3 & 1288.40Hz \\
            4 & 1288.41 Hz \\
            6 & 1288.00 Hz \\
        \bottomrule
    \end{tabularx}
    \caption{Frequenzen der Testmessungen}
    \label{tab:mytable}
\end{table}

Wir gehen davon aus, das der systematische Fehler symmetrisch verteilt ist.
Zum Abschätzen betrachten wir die größte Abweichung einer Einzelnen Fehler Messung vom Erwartungswert und nehmen diese dann als Toleranz.
Da der Erwartungswert bei $1288.22 Hz$ liegt, hat die Messung 6 die Größte Abweichung von $0.22 Hz$.

Somit kennen wir die Systematischen Fehler auf alle unsere Größen.\\

%TODO : Systematischer Fehler auf Frequenz bestimmen

Tabelle der Systematischen Fehler auf einzelne Messgrößen: 

\begin{table}[H]
    \centering
    \begin{tabularx}{0.8\textwidth}{X l} % adjust width as needed
        \toprule
        \textbf{Metall Stange} & \textbf{Masse} \\
        \midrule
        Masse & $\pm$0.2g \\
        Länge & $\pm$0.7mm\\
        Durchmesser & $\pm$0.01mm \\
        Frequenz & $\pm$0.423mV (Digitalisierung)\\
        & $\pm$0.22Hz (Einspannung) \\
        \bottomrule
    \end{tabularx}
    \label{tab:mytable}
\end{table}

Da der Digitalisierungsfehler auf die Frequenz einen deutlich kleineren Einfluss hat als der Einspannungsfehler, haben wir den Digitalisierungsfehler auf die Frequenz in der folgenden Fehlerbetrachtung vernachlässigt.

\begin{figure}[H]
  \centering
    \caption{Systematischer Fehler auf die Frequenz}
    \subfloat{\includegraphics[width=0.5\textwidth]{plots/434170_428396_1A3_frequenzen_stat_err_Stahl.pdf}}
  \hfill
    \subfloat{\includegraphics[width=0.5\textwidth]{plots/434170_428396_1A3_frequenzen_stat_err_Aluminium.pdf}}
\end{figure}
\begin{figure}[H]
  \centering
    \subfloat{\includegraphics[width=0.5\textwidth]{plots/434170_428396_1A3_frequenzen_stat_err_Messing.pdf}}
  \hfill
    \subfloat{\includegraphics[width=0.5\textwidth]{plots/434170_428396_1A3_frequenzen_stat_err_Kupfer.pdf}}
\end{figure}
 
Aus den oben angeführten Frequenz haben wir dann mit Hilfe eines Python Programm (Mat1Mat2Statistik) das Arithmetische mittel und daraus den Resultierenden statistischen Fehler berechnet. Hierfür haben wir folgende Beziehungen für Statistische Messunsicherheiten verwendet. Wobei $\mu$ im allgemeinen der Mittelwert ist, $s^2$ die Varianz und $\sigma_i$ der Fehler auf die Größe ist.

\begin{equation}
	\mu = \frac{1}{n}\sum_{i=1}^nx_i
\end{equation}

\begin{equation}
	s^2 = \frac{1}{n-1}\sum_{i=1}^n(\mu-x_i)^2
\end{equation}

\begin{equation}
	\sigma_f = \frac{s}{\sqrt{n}} 
\end{equation}

Für die verschiedenen Stangen ergaben sich dann Folgende Mittelwerte und statistische Fehler.\\


 \begin{table}[H]
        \centering
        \begin{tabularx}{1\textwidth}{X X X X} % adjust width as needed
            \toprule
            \textbf{Metall Stange} & \textbf{Erwartungswert der Frequenz} & \textbf{stat. Fehler auf Frequenz} & \textbf{syst. Fehler auf Frequenz} \\
            \midrule
            Stahl 15 & 1726.53Hz & $\pm$0.094Hz & $\pm$0.2Hz\\
            Aluminium & 1673.66Hz & $\pm$0.08Hz & $\pm$0.2Hz \\
            Messing & 1166.40Hz & $\pm$0.052Hz & $\pm$0.2Hz \\
            Kupfer & 1288.22Hz & $\pm$0.10Hz & $\pm$0.2Hz \\

            \bottomrule
        \end{tabularx}
        \label{tab:mytable}
    \end{table}
    
    % TODO: add python program name
Als weiteren Statistischen Fehler haben wer den Fehler der bei der mehrfachen Messung des Durchmessers der Stangen auftritt.
Wir haben diese mehrfach gemessen, um zu überprüfen ob unsere Annahme, dass die Stange einen 
Kreisförmigen Querschnitt hat gerechtfertigt ist. Zur Bestimmung des Mittelwerts und Fehlers auf den Durchmesser 
haben wir ebenfalls das Pythonprogramm ... benutzt. 
Daraus ergibt sich:\\

 \begin{table}[H]
        \centering
        \begin{tabularx}{1.0\textwidth}{X c c} % adjust width as needed
            \toprule
            \textbf{Metall Stange} & \textbf{Erwartungswert auf den Durchmesser} & \textbf{stat. Fehler auf Durchmesser} \\
            \midrule
            Stahl 15 & 12.000mm & $\pm$0.001mm \\
            Aluminium & 12.057mm & $\pm$0.002mm  \\
            Messing & 11.986mm & $\pm$0.003mm \\
            Kupfer & 11.981mm & $\pm$0.001mm \\
            \bottomrule
        \end{tabularx}
        \label{tab:mytable}
    \end{table}

Somit haben wir den Messfehler auf alle Größen bestimmt und können damit das Elastizitätsmodule berechnen. Hierbei müssen die Statistischen und die Systematischen Fehler getrennt fortgepflanzt werden.

Für die Fehlerfortpflanzung der Statistischen Fehler gilt: (hier wird die Gaußsche Fehlerfortpflanzung verwendet) 

\begin{equation}
	\sigma_y^2 = \sum_{i,j=1}^n\left[\frac{\partial y}{\partial x_i}\frac{\partial y}{\partial x_j}\right]_{x=\mu}V_{ij}^2
\end{equation}
$V_{ij}$ ist die Varianz zwischen denn Variablen. In unserem Fall sind die Variablen die Masse, Länge, Grenzfrequenz und Durchmesser. 
Bei diesen können wir davon ausgehen, dass diese Größen unkorreliert sind. 
Also unsere Längenmessung keinen Einfluss auf die Messung des Durchmessers und der Masse hatte.
Das selbe gilt für die anderen Messgrößen. 
Somit müssen wir den Mischterm auf Grund der Kovarianz nicht beachten da diese null ist.
Somit vereinfacht sich die Gauß'sche Fehlerfortpflanzung zu:

\begin{equation}
	\sigma_y^2 = \sum_{i=1}^n\left[\frac{\partial y}{\partial x_i}\right]_{x=\mu}\sigma_{i}^2
\end{equation}

Bei den Systematischen Fehlern verwenden wir die Methode der Größtfehlerabschätzung. 
Dies ist zwar eine sehr konservative Schätzung des Fehlers, in diesem Fall jedoch sinnvoll,
da Masse und Länge je nur einmal gemessen wurden. wenn nun $\Delta x_1$, $\Delta x_2$, $\Delta x_3$
die eingehenden Einzelfehler sind, ist die lineare Fehlerfortpflanzung: \\

\begin{equation}
	\Delta y = \left|\frac{\partial y}{\partial x_1}\Delta x_1\right| + 
	\left|\frac{\partial y}{\partial x_2}\Delta x_2\right| + 
	\left|\frac{\partial y}{\partial x_3}\Delta x_3\right|
\end{equation}\\

Hier gehen wir ebenfalls davon aus, dass die verschiedenen Größen unkorreliert sind. 

Als Zwischenergebnis berechnen wir zuerst die Dichte. Hierfür berechnen wir den
Erwartungswert der Dichte indem wir die Mittelwerte der Benötigten Größen Verwenden. 
Wir verwenden die Formel die in dem Abschnitt Grundlagen eingeführt werden. Die Rechnung haben wir mit Python durchgeführt. Diese können sie in der Datei ... nachlesen.\\

\begin{equation}
    \text{Formel für die Dichte: }\qquad \rho = \frac{M}{V} = \frac{4M}{\pi D^2L}
\end{equation}\\

Zusätzlich dazu haben wir noch systematische Fehler. Für unser spezifisches Problem lautet die Rechnung die wir machen:

\begin{equation}
	\Delta_{\rho} = \left|\frac{-8M}{\pi D^3L}\Delta_D\right| + 
	\left|\frac{4}{\pi D^2L}\Delta_M\right| + 
	\left|\frac{-4M}{\pi D^2L^2}\Delta_L\right|
\end{equation}\\

Wobei Hier für M, L, D der Mittelwert bzw. der beste schätzwert für die Größen eingesetzt wird, abhängig 
vom Material. Und $\Delta_i$ für die Systematischen Fehler auf die einzelnen Größen steht. 
Diese Fehler können sie der Tabelle XYZ entnehmen. 
Dies Gleichung haben wir ebenfalls mit einem Python Skript (...)
ausgewertet.

Hieraus ergeben sich folgende Erwartungswert und Fehler auf die Dichte. 


 \begin{table}[H]
        \centering
        \begin{tabularx}{1\textwidth}{X X X X} % adjust width as needed
            \toprule
            \textbf{Metall Stange} & \textbf{Erwartungs Wert Dichte [$\frac{kg}{m^3•}$]} & \textbf{stat. Fehler auf die Dichte [$\frac{kg}{m^3•}$]} & \textbf{syst. Fehler auf die Dichte [$\frac{kg}{m^3•}$]}\\
            \midrule
            Stahl 15 & 7824.53 & $\pm$1.94 & $\pm$4.23 \\
            Aluminium & 2691.21 & $\pm$0.95 & $\pm$2.62 \\
            Messing & 8434.25 & $\pm$4.30 & $\pm$4.38 \\
            Kupfer & 8901.94 & $\pm$1.49 & $\pm$4.49\\
            \bottomrule
        \end{tabularx}
        \label{tab:mytable}
    \end{table}

Dieser Tabelle können Sie entnehmen, dass der statistische und der Systematischen Fehler auf die Dichte in der gleichen Größenordnung liegen. 

Um zu veranschaulichen welche Größen, den Größten Einfluss auf dieses Zwischenergebnis haben, hier noch ein mal eine Tabelle mit den Systematischen 
und Statistischen Fehlern auf die Einzelgrößen. 

 \begin{table}[H]
        \centering
        \begin{tabularx}{1\textwidth}{l X X X X} % adjust width as needed
            \toprule
            \textbf{Stange} & \textbf{Aluminium} & \textbf{Messing} & \textbf{Kupfer} & \textbf{Stahl} \\
            \midrule
            Syst. Fehler auf M & $\pm$0.0002kg & $\pm$0.0002kg & $\pm$0.0002kg & $\pm$0.0002kg \\
            Syst. Fehler auf L & $\pm$0.0007m & $\pm$0.0007m & $\pm$0.0007m & $\pm$0.0007m \\
            Syst. Fehler auf D & $\pm$0.00001m & $\pm$0.00001m & $\pm$0.00001m & $\pm$0.00001m \\

            \midrule
            Stat. Fehler auf M & 0 & 0 & 0 & 0 \\
            Stst. Fehler auf L & 0 & 0 & 0 & 0 \\
            Stst. Fehler auf D & $\pm$0.000002m & $\pm$0.000003m & $\pm$0.000001m & $\pm$0.000001m \\
           
            \bottomrule
        \end{tabularx}
        \label{tab:mytable}
    \end{table}

In dieser Tabelle sind auch sämtliche eingegangenen Fehler bereits in SI Einheiten dargestellt. Somit kann auf den ersten Blick gesehen werden, welche Fehler den größeren 
Einfluss haben. Wir können wir sehen, dass der gesamt systematische Fehler um 2 Größenordnungen größer ist als jener des statistischen Fehlers. Wenn wir nun exemplarisch den Fehler auf den Durchmesser von Aluminium anschauen.
 D = (12.057 $\pm$0.002 (stat), $\pm$0.01 (syst))mm 
 
Erkennen wir, dass die relative Abweichung aufgrund des statistischen 
Fehlers nur 0.017\% entspricht. während die auf Grund des Systematischen Fehlers
0.083\% entspricht. Somit hat der systematische Fehler des Durchmessers
einen größeren Einfluss auf das Endergebnis. Wobei der Systematische Fehler auf Masse 
Masse und Länge jeweils größer ist als jener auf den Durchmesser. 
Hier darf aber nicht vergessen werden, dass wir den Systematischen Fehler linear fortgepflanzt haben was zu einer sehr konservativen Schätzung führt. Dies erhöht 
natürlich den Einfluss der systematischen Fehler auf das Endergebnis. 

Des weiteren kann man an der Tatsache, dass das statistische Fehler auf den Durchmesser 
signifikant kleiner ist als der Systematische erkennen, dass unser Annahme, 
dass die Stangen zylinderförmig sind eine gerechtfertigte Annahme war. 

Im gesamten können wir also sagen, dass die Systematischen Fehler auf M L und D, den
Fehler auf die Dichte in Endeffekt dominiert haben. Wobei der Systematische Fehler auf
die Länge den größten Einfluss hatte.  

Mit der Dichte können wir nun auch das Elastizitätsmodule bestimmen. Dies wird mit der folgenden Formel berechnet. 

    \begin{equation}
         E = (2fL)^2 \rho
    \end{equation}
    
Hier müssen die Systematischen und die Statistischen Fehler ebenfalls getrennt fortgepflanzt werden. Wobei hier ebenfalls gilt, dass die einzelnen Messgrößen 
unkorreliert sind. Hierfür haben wir das selbe vorgehen wie bei der Dichte gewählt.
Wir können also die Selben Formeln (14) (15) verwenden. 
Wobei dieses mal für die Funktion y die Formel für das Elastizitätsmodule verwendet wird. Die Berechnung haben wir ebenfalls mit dem Pythonprogramm durchgeführt. 
Die Rechnung haben wir mit ungerundeten Werten durchgeführt, damit Rundungsfehler nur einen kleinen Einfluss auf die Rechnung haben. 
Dies Sollte auch dafür sorgen, dass das Ergebnis sich nicht verändert wenn wir Formel (18) oder Formel (6) verwenden.
Die daraus folgenden Ergebnisse können sie der Unten stehenden Tabelle entnehmen mit dem relativen Fehler in Klammern.

 \begin{table}[H]
        \centering
        \begin{tabularx}{1\textwidth}{X X X X} % adjust width as needed
            \toprule
            \textbf{Metall Stange} & \textbf{Erwartungswert E-Modul [$GPa$]} & \textbf{stat. Fehler auf E-Modul [$GPa$]} & \textbf{syst. Fehler auf E-Modul [$GPa$]}\\
            \midrule
            Stahl 15 & 209.92 & $\pm$0.29  (0.1\%) & $\pm$6.11 (2.9\%) \\
            Aluminium & 67.85 & $\pm$0.104 (0.2\%) & $\pm$2.28 (3\%) \\
            Messing & 103.27  & $\pm$0.198 (0.2\%) & $\pm$3.24 (3\%) \\
            Kupfer & 132.96   & $\pm$0.22  (0.2\%) & $\pm$4.05 (3\%) \\
            \bottomrule
        \end{tabularx}
        \label{tab:mytable}
    \end{table}

Zur übersichtlichen Diskussion der Fehler und deren Beitrag zu dem Endgültigen Ergebnis stellen wir Systematische und Statistische Fehler tabellarisch gegenüber.
Hierbei ist f die Frequenz und $\rho$ die Dichte.

 \begin{table}[H]
        \centering
        \begin{tabularx}{1\textwidth}{l X X X X} % adjust width as needed
            \toprule
            %\textbf{Metall Stange} & \textbf{Erwartungswert der Frequenz} & \textbf{stat. Fehler auf die Frequenz} & \textbf{syst. Fehler auf die Frequenz} \\
            \textbf{Metall Stange} & \textbf{Aluminium} & \textbf{Messing} & \textbf{Kupfer} & \textbf{Stahl 15} \\
            \midrule
            Erwartungswert von f & 1673.66Hz & 1166.40Hz & 1288.22Hz & 1726.53Hz \\
            stat. Fehler von f & $\pm$0.08Hz & $\pm$0.052Hz & $\pm$0.10Hz & $\pm$0.094Hz \\
            syst. Fehler von f & $\pm$0.2Hz & $\pm$0.2Hz & $\pm$0.2Hz & $\pm$0.2Hz \\
            \midrule
            stat. Fehler von $\rho$ & $\pm$0.95$kg/m^3$ & $\pm$4.30$kg/m^3$ & $\pm$1.49$kg/m^3$ & $\pm$1.94$kg/m^3$ \\
            syst. Fehler von $\rho$ & $\pm$2.62$kg/m^3$ & $\pm$4.38$kg/m^3$ & $\pm$4.49$kg/m^3$ & $\pm$4.23$kg/m^3$ \\
            \midrule
            syst. Fehler von L      & $\pm$0.00007m     & $\pm$0.00007m     & $\pm$0.00007m     & $\pm$0.00007m \\
            stat. Fehler von L      & $\pm$0m     & $\pm$0m     & $\pm$0m     & $\pm$0m \\
            \bottomrule
        \end{tabularx}
    \end{table}
Wie zu sehen ist, ist der systematische Fehler auf die Frequenz deutlich größer als der statistische.
Eine Mögliche Ursache dafür ist, das wir bei der Fehlerabschätzung für den statistischen Fehler bezüglich der Einspannposition eine sehr konservative Abschätzung gemacht haben,
da wir den Wert mit der größten Abweichung vom Erwartungswert genommen haben.

Dies bedeutet das der größte Einfluss auf den Fehler der von der Frequenz stammt, aufgrund der Einspannung ist.
Da wir bei der Durchführung der Messreihe die Einspannposition nicht verändert haben, ist der statistische Messfehler davon unbeeinflusst.

Bezüglich der Dichte sind Statistischer und Systematischer Fehler ca. gleich groß. \\
 
Zur Veranschaulichung, ob der systematische Fehler der Dichte oder der Frequenz den größten Einfluss auf das Elastizitätsmodule hat, haben wir von beiden Fehlerquellen getrennt die relative Abweichung berechnet.
Beispielhaft für Aluminium:
$ \rho_{Aluminium} = 9.8 \quad 10^{-4} $ und $ f_{Aluminium} = 1.2 \quad 10^{-4} $
Die relative Abweichung der Dichte ist größer als die der Frequenz. Dies ist auch bei allen anderen Stäben der Fall.

Die systematische Fehler der Längenmessung ist sehr klein gegenüber den anderen Fehlern.

Da in der Tabelle alle Werte in SI Einheiten dargestellt sind, kann man direkt ablesen,
das bei der Fehlerfortpflanzung auf das E-Modul die Dichte den größten Einfluss hat. 
Das bedeutet das sowohl der Systematischer wie der statistische Fehler am stärksten von der Dichte beeinflusst wurden.
Das heißt, das für den Statistischen Messfehler der größte Einfluss auf das E-Modul von dem Messen des Durchmessers der Stäbe mit der Millimeter Schraube kommt.
 
Das E-Modul hat eine relative Abweichung von ca. 3\% beim Systematischen Fehler bzw. ca. 0.15\% für den Statistischen Messfehler.
Das bedeutet das die Systematischen Messfehler deutlich überwiegen.
 
Der Statistische Messfehler auf das E-Modul wurde am stärksten von der Längenmessung beeinflusst, da diese den größten Einfluss auf den systematischen Fehler der Dichte hatte. \\
 
\textbf{Vergleich mit der Literatur} \\
Für die Kupferstange haben wir mit Hilfe der Datenbank (matweb.com) herausgefunden, das es 4 Kupfer Legierungen gibt (z.B Copper Alloy, UNS C64760, H06 Temper, Flat Products), welche ein in etwa passendes E-Modul.
Das E-Modul der Legierung ist 132$GPa$, welches innerhalb der Messgenauigkeit unseres Ergebnisse liegt.
Jedoch stimmt hier die dichte von 8800$kg/m^3$ nicht mit unserem gemessenen Wert überein.
 
ür das Material (Oxygen-free high conductivity Copper, Hard, UNS C10200) passen sowohl die Dichte (8900$kg/m^3$) als auch das E-Modul (117-132 $GPa$) mit unseren gemessenen Werten im Rahmen der Unsicherheit überein. \\
 
Bei Aluminium gab es sehr viele Treffer von Aluminium (z.B Aluminium 6463-T6).
Diese haben alle ein E-Modul von 68.9 - 69.0 $GPa$, welches etwas über dem von uns bestimmten Wert liegt, aber innerhalb der Unsicherheiten.
Die Dichte aus der Datenbank von Aluminium ist ca 2690$kg/m^3$ entspricht exakt unserem Wert. \\
 
 
Im Fall von Stahl ist der Vergleich mit Literatur Werten etwas schwieriger. 
Die Datenbank listet viele Stahl Legierungen auf welche auf unsere Messungen zutreffen.
Diese sind meist mit einem großen Intervall angegeben, in welchem das Elastizitätsmodule liegt.
Somit könnte unsere Stahl Stange aus der Gruppe (Overview of materials for AISI 5000 Series Steel) von Stahl stammen. Diese haben einen Bereich des E-Moduls von 200-210 $GPa$ und eine Dichte von 7.80-7.85$kg/m^3$. Hier Liegt unser Stahl am oberen Ende des Intervalls des E-Moduls. Die Unterschiede liegen jedoch noch im Rahnem der Messgenauigkeit. 
Am besten zutreffen auf unsere Werte würde das Material (Bohler-Uddeholm BÖHLER M261 Extra Plastic Lens Mold Steel). Dieses hat ein E-Modul von 210 $GPa$ und eine Dichte von
7.82$kg/m^3$. Hier stimmt vor allem die Dichte sehr gut mit unserer Berechnung überein.\\
 
Für Messing sind die passenden Legierungen alle Copper-Beryllium Legierungen(z.B Materion Beryllium Copper Alloy 165 Forgings \& Extrusions).
Diese haben exakt die passende Dichte und ein E-Modul von 131 $GPa$, welches innerhalb des systematischen Messfehlers liegt.
 
Im gesamten können wir sagen, dass man mithilfe der Eigenschaften die wir von den Materialien bestimmt haben das genaue Material auf eine kleine Menge von Möglichkeiten einschränken kann. 
Es reicht jedoch nicht aus, um das genaue Material zu bestimmen. Hierfür müssten wir 
eine weitere Eigenschaft bestimmen. Die Verringerung der Messfehler würde nicht reichen
um das Material weiter ein zu schränken, da die Werte in der Datenbank meist nicht so genau angegeben sind. 
\end{aufgabe}



%TODO : Diskussioon der Fehlerbeiträge, was dominiert den Gesamtfehler
%TODO : Gesamtprotokoll auf Konsistenz überprüfen in Bezeichnung 
%TODO : Verwendete Zeichen in Formeln etc. Erklärt
%TODO : Statistische Fehler mit Sigma Systematische mit Delta
%TODO : Konsistenz von Standardabweichung und Fehler bezeichnung
%TODO : Tabellen und Bilder eindeutig benennen
%TODO : Zwischenüberschriften 
\end{document}

 
\end{document}
